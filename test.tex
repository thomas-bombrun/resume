\documentclass{article}           %% ceci est un commentaire (apres le caractere %)
\usepackage[latin1]{inputenc}     %% adapte le style article aux conventions francophones
\usepackage[T1]{fontenc}          %% permet d'utiliser les caractères accentués
\usepackage[dvips]{graphicx}      %% permet d'importer des graphiques au format .EPS (postscript)
\usepackage{fancybox}		   %% package utiliser pour avoir un encadré 3D des images
\usepackage{makeidx}              %% permet de générer un index automatiquement
\title{Un exemple de ce que l'on peut faire}     %% \title est une macro, entre { } figure son premier argument
\author{Kévin JASON}        %% idem

\makeindex		    %% macro qui permet de générer l'index
\bibliographystyle{prsty}	  %% le style utilisé pour créer la bibliographie
\begin{document}                  %% signale le début du document

\maketitle                        %% produire à cet endroit le titre de l'article à partir des informations fournies ci-dessus (title, author)
\newpage
\tableofcontents                  %% produire à cet endroit la table des matièree					
\newpage
\section{Exemple de fonction mathématique}            %% un titre de niveau 1 qui sera inclus dans la table des matières
Quelle que soit la valeur de $x$, 
la propriété suivante est toujours 
vérifiée: $$\sin^2 x+\cos^2 x=1$$ 
On peut s'en douter en observant 
le tracé de la fonction illustrée 
dans la figure~\ref{courbe} 
à la page~\pageref{courbe}.
\index{fonction}                  %% inclure le mot fonction dans l'index
\newpage
\subsection{Un premier exemple d'image}
\begin{figure}[htbp]
\centering
\includegraphics[width=\textwidth]{courbe.eps}
\caption{Cette figure illustre 
le tracé de la fonction 
$f(x)\equiv\sin^2 x+\cos^2 x=1$.}
\label{courbe}
\index{tracé}                     %% inclure le mot tracé dans l'index
\index{fonction}                  %% include le mot fonction dans l'index
\end{figure}
\newpage
\subsection{Un deuxième exemple d'image}
\begin{figure}[htbp]
\centering
\shadowbox{ \begin{minipage}{10cm}
\centering
\includegraphics[height=4cm]{courbe.eps}
\caption{\footnotesize courbe.eps}
\end{minipage} }
\end{figure}

\section{Un exemple avec référence bibliographique}
The discovery of the Quantized Hall Effect\footnote{je sais pas ce que c'est!} was made by
Klitzing~\cite{qhe} for which he was awarded the 1985 Nobel
prize for physics~\cite{nobel}.

\section{Un exemple de tableau}
\begin{tabular}{|*{2}{c|}l r|}
   \hline
   une & deux & trois & quatre \\
   case centrée & encore centrée & à gauche & à droite \\
   \hline
   six & sept & huit & neuf \\
   \hline
\end{tabular}
\section{Conclusion}               %% un autre titre
\index{conclusion}                 %% inclure le mot conclusion dans l'index
\index{LaTeX}                      %% inclure le mot LaTeX dans l'index
En conclusion, \LaTeX\ est 
particulièrement bien adapté pour 
rédiger de longs documents.
 											 
\input DemoLaTeX.ind               %% inclure ici l'index produit à partir de ce document
\bibliography{DemoLaTeX}

\end{document} 
