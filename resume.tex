%%%%%%%%%%%%%%%%%%%%%%%%%%%%%%%%%%%%%%%%%
% "ModernCV" CV and Cover Letterpo
% LaTeX Template
% Version 1.1 (9/12/12)
%
% This template has been downloaded from:
% http://www.LaTeXTemplates.com
%
% Original author:
% Xavier Danaux (xdanaux@gmail.com)
%
% License:
% CC BY-NC-SA 3.0 (http://creativecommons.org/licenses/by-nc-sa/3.0/)
%
% Important note:
% This template requires the moderncv.cls and .sty files to be in the same 
% directory as this .tex file. These files provide the resume style and themes 
% used for structuring the document.
%
%%%%%%%%%%%%%%%%%%%%%%%%%%%%%%%%%%%%%%%%%

%----------------------------------------------------------------------------------------
%	PACKAGES AND OTHER DOCUMENT CONFIGURATIONS
%----------------------------------------------------------------------------------------


\documentclass[11pt,a4paper,sans]{moderncv} % Font sizes: 10, 11, or 12; paper sizes: a4paper, letterpaper, a5paper, legalpaper, executivepaper or landscape; font families: sans or roman
\usepackage[utf8]{inputenc}
\usepackage{lmodern}


\moderncvstyle{casual} % CV theme - options include: 'casual' (default), 'classic', 'oldstyle' and 'banking'
\moderncvcolor{blue} % CV color - options include: 'blue' (default), 'orange', 'green', 'red', 'purple', 'grey' and 'black'


\usepackage[scale=0.9]{geometry} % Reduce document margins
%\setlength{\hintscolumnwidth}{2.25cm} % Uncomment to change the width of the dates column
%setlength{\makecvtitlenamewidth}{10cm} % For the 'classic' style, uncomment to adjust the width of the space allocated to your name
\usepackage{xpatch}
\xpatchcmd{\cventry}{.\strut}{\strut}{}{}

%----------------------------------------------------------------------------------------
%	NAME AND CONTACT INFORMATION SECTION
%----------------------------------------------------------------------------------------

\firstname{} % Your first name
\familyname{Thomas Bombrun} % Your last name

% All information in this block is optional, comment out any lines you don't need
\mobile{06 95 43 85 92}
\email{thomas.bombrun@gmail.com}
\title{}
%----------------------------------------------------------------------------------------
\begin{document}
\makecvtitle % Print the CV title
\vspace*{-1cm}

%----------------------------------------------------------------------------------------
%	EDUCATION SECTION
%----------------------------------------------------------------------------------------
\section{Formation}
\cventry
    {2012--2016}
    {Diplôme d'Ingénieur en Informatique et Mathématiques appliquées}
    {}
    {Ensimag}
    {Grenoble}
    {Filière Ingénierie des Systèmes d'Information}
    
\cventry
    {2010--2012}
    {L2 Mathématiques et Informatique}
    {}
    {Université Joseph Fourier}
    {Grenoble}
    {Section Internationale}
    
\cventry
    {2010}
    {Baccalauréat Scientifique}
    {}
    {Lycée François-Jean Armorin}
    {Drôme}
    {Mention Bien}

%----------------------------------------------------------------------------------------
%	WORK EXPERIENCE SECTION
%----------------------------------------------------------------------------------------

\section{Expériences}

\subsection{Professionnelles}


\cventry
    {Depuis Octobre 2016}
    {Ingénieur Web Fullstack}
    {}
    {Cirruseo}
    {Paris}
    {
        \begin{itemize}
            \item Développement d'applications web basées sur l'environnement Google : Google Cloud Platform et GSuite.
            \item Depuis Mai 2017 : Lead Developer
            \begin{itemize}
                \item Responsable du backlog des projets
                \item Suivi technique des équipes de développeur.
                \item Référent technique Python et Google Cloud Platform.
            \end{itemize}
        \end{itemize}
        \textit{Python, Google Cloud Platform, Google App Engine, Polymer}
        \begin{itemize}
            \item Depuis Octobre 2017 : Ingénieur DevOps auprès d'Airbus Defence and Space
                \begin{itemize}
                \item Développement d'un backend Go de cataloguage d'images satellites.
                \item Déploiement de la solution sur différentes solutions cloud.
            \end{itemize}
        \end{itemize}
        \textit{Go, Docker, Kubernetes, Helm, Elasticsearch}
    }

\cventry
    {2015}
    {Projet de Fin d'Études}
    {5 mois}
    {Tessi Lab}
    {Grenoble}
    {
        \begin{itemize}
            \item Conception et implémentation d'une solution de lecture et d'analyse automatique de documents papier photographiés. Cette solution a permis d'automatiser la vérification manuelle de documents au sein de la branche marketing de Tessi.
            \item Responsable de l'architecture logicielle.
        \end{itemize}
        \vspace*{1mm}
        \textit{Java, Elasticsearch, Traitement d'images, Reconnaissance optique de caractères (Tesseract), Théorie des langages.}
    }

\cventry
    {2014}
    {Stage Assistant Ingénieur}
    {2 mois}
    {Sharette}
    {Paris}
    {
        \begin{itemize}
            \item Développement d'une application mobile de covoiturage.
        \item Implication dans les prises de décisions de cette start-up.
        \end{itemize}
        \vspace*{1mm}
        \textit{HTML5, AngularJS, PhoneGap}
    }

\subsection{Académiques}

\cventry
    {2016}
    {Projet Système Distribué}
    {}
    {Ensimag}
    {}
    {
        \begin{itemize}
            \item Développement d'une version distribuée de GNU Make.
            \item Analyse de performances sur un algorithme distribué.
        \end{itemize}
        \vspace*{1mm}
        \textit{Scala, Akka, Grille de calcul}
    }

%\cventry
%   {2016}
%   {Projet d'Infrastructure Logicielle pour le Bâtiment Intelligent}
%   {}
%   {Ensimag}
%   {}
%   {
%       \begin{itemize}
%           \item Réalisation d'une maquette de démonstration domotique.
%       \end{itemize}
%       \vspace*{1mm}
%       \textit{OpenHab, Arduino, ESP8266, MQTT}
%   }

\cventry
    {2016}
    {Challenge Open Data}
    {}
    {Ensimag}
    {}
    {
        \begin{itemize}
            \item Production d'une visualisation interactive des similarités les entre députés via l'analyse des comptes rendus de vote de l'Assemblée Nationale.
        \end{itemize}
        \vspace*{1mm}
        \textit{HTML5 Canvas, PostgreSQL, Python}
    }

%\cventry
%   {2014}
%   {Projet Web}
%   {}
%   {Ensimag}
%   {}
%   {
%       \begin{itemize}
%           \item Conception et développement d'un jeu web pour le CCSTI de Grenoble.
%       \end{itemize}
%       \vspace*{1mm}
%       \textit{JavaScript, Phaser.}
%   }


% \subsection{Associatives}

% \cventry
%   {2013--2014}
%   {Membre du Bureau des étudiants}
%   {}
%   {Ensimag}
%   {}
%   {}

% \cventry
%   {}
%   {Chef d'équipe au sein du festival Crest Jazz Vocal}
%   {}
%   {Drôme}
%   {}
%   {}

%----------------------------------------------------------------------------------------
%	COMPUTER SKILLS SECTION
%----------------------------------------------------------------------------------------

\section{Compétences}

\cvdoubleitem
    {Langages}
    {Python, Java, C, C\texttt{++}, Scala, ADA}
    {Autres}
    {Bash, \LaTeX, Unix, Docker, Google Cloud Platform}
\cvdoubleitem
    {Web}
    {HTML5, CSS, JavaScript, Polymer, AngularJS, React, Node.js}
    {Base de données}
    {Oracle SQL, PostgreSQL, MongoDB, Elasticsearch, Google Cloud Datastore}

% \cvitem
%     {}
%     {Capable et prêt à monter en compétences sur toutes technologies ou langages.}

%----------------------------------------------------------------------------------------
%	LANGUAGES SECTION
%----------------------------------------------------------------------------------------

\section{Langues}

\cvitemwithcomment
    {}
    {Anglais : Courant}
    {990/990 au TOEIC (décembre 2013)}


\end{document}
