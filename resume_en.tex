%%%%%%%%%%%%%%%%%%%%%%%%%%%%%%%%%%%%%%%%%
% "ModernCV" CV and Cover Letterpo
% LaTeX Template
% Version 1.1 (9/12/12)
%
% This template has been downloaded from:
% http://www.LaTeXTemplates.com
%
% Original author:
% Xavier Danaux (xdanaux@gmail.com)
%
% License:
% CC BY-NC-SA 3.0 (http://creativecommons.org/licenses/by-nc-sa/3.0/)
%
% Important note:
% This template requires the moderncv.cls and .sty files to be in the same 
% directory as this .tex file. These files provide the resume style and themes 
% used for structuring the document.
%
%%%%%%%%%%%%%%%%%%%%%%%%%%%%%%%%%%%%%%%%%

%----------------------------------------------------------------------------------------
%	PACKAGES AND OTHER DOCUMENT CONFIGURATIONS
%----------------------------------------------------------------------------------------


\documentclass[11pt,a4paper,sans]{moderncv} % Font sizes: 10, 11, or 12; paper sizes: a4paper, letterpaper, a5paper, legalpaper, executivepaper or landscape; font families: sans or roman
\usepackage[utf8]{inputenc}
\usepackage{lmodern}


\moderncvstyle{casual} % CV theme - options include: 'casual' (default), 'classic', 'oldstyle' and 'banking'
\moderncvcolor{blue} % CV color - options include: 'blue' (default), 'orange', 'green', 'red', 'purple', 'grey' and 'black'


\usepackage[scale=0.9]{geometry} % Reduce document margins
%\setlength{\hintscolumnwidth}{2.25cm} % Uncomment to change the width of the dates column
%setlength{\makecvtitlenamewidth}{10cm} % For the 'classic' style, uncomment to adjust the width of the space allocated to your name
\usepackage{xpatch}
\xpatchcmd{\cventry}{.\strut}{\strut}{}{}

%----------------------------------------------------------------------------------------
%	NAME AND CONTACT INFORMATION SECTION
%----------------------------------------------------------------------------------------

\firstname{} % Your first name
\familyname{Thomas Bombrun} % Your last name

\title{Software Engineer} 

% All information in this block is optional, comment out any lines you don't need
\mobile{+336 95 43 85 92}
\email{thomas.bombrun@gmail.com}
\title{}
%\extrainfo{\href{http://github.com/thomas-bombrun/resume}{github.com/thomas-bombrun/resume}} % When bragging is in order

%----------------------------------------------------------------------------------------
\begin{document}
\makecvtitle % Print the CV title
\vspace*{-0.75cm}

%----------------------------------------------------------------------------------------
%	WORK EXPERIENCE SECTION
%----------------------------------------------------------------------------------------

\section{Work experience}

\cventry
    {2018-2022}
    {Backend Engineer}
    {}
    {Airbus Defence and Space}
    {Toulouse}
    {
        \begin{itemize}
            \item Development of a satellite image catalog backend in Go (orchestration of the daily ingestion of satellite images, cataloging of satellite image metadata, access rights management, implementation of Open Geospatial Consortium visualization standards)
            \item Micro-services architecture. Running on Kubernetes, using RabbitMQ, leveraging Elasticsearch for indexation of satellite image metadata
            \item Multi-cloud deployments (Google Cloud Platform and on-premises deployments using Openstack)
            \item Since January 2020 : \textbf{Lead Developer}
            \begin{itemize}
                \item Feature review with product owners and software architects
                \item Code review with developers
                \item Software development, support for operations team, level 3 support
            \end{itemize}
        \end{itemize}
        \textit{Go, Docker, Kubernetes, Helm, Elasticsearch, PostgreSQL, RabbitMQ, Google Cloud Platform, OpenStack}
    }
\vspace*{0.2cm}


\cventry
    {2016-2018}
    {Full-Stack Engineer}
    {}
    {Cirruseo}
    {Paris}
    {
        \begin{itemize}
            \item Web application development based around Google technologies : Google Cloud Platform and Google Workspace.
            \item Since May 2017 : Lead Developer
            \begin{itemize}
                \item Sprint planning with project managers.
                \item Technical help to other developers.
                \item Python and Google Cloud Platform specialist.
            \end{itemize}
            \item September 2017 : Long term assignment with Airbus Defence and Space (Toulouse)
        \end{itemize}
        \textit{Python, Google Cloud Platform, Google App Engine, Polymer}
    }
\vspace*{0.2cm}


%----------------------------------------------------------------------------------------
%	EDUCATION SECTION
%----------------------------------------------------------------------------------------
\section{Education}
\cventry
    {2012--2016}
    {\textit{Titre d’ingénieur} in computer science and applied mathematics }
    {}
    {Ensimag}
    {Grenoble, France}
    {Specialization : Information systems engineering}
    
\cventry
    {2010--2012}
    {2 years of Bachelor in Mathematics and Computer Science}
    {}
    {Joseph Fourier University}
    {Grenoble, France}
    {International Section}
    
\cventry
    {2010}
    {French Scientific Baccalaureate}
    {}
    {Lycée François-Jean Armorin}
    {Crest, Drôme, France}
    {With high honours}

%----------------------------------------------------------------------------------------
%	COMPUTER SKILLS SECTION
%----------------------------------------------------------------------------------------

\section{Skills}

\cvline{\footnotesize Languages}{Python, Go, Java, C, C\texttt{++}, Scala, ADA}
\cvline{\footnotesize Databases}{Oracle SQL, PostgreSQL, MongoDB, Elasticsearch, Google Cloud Datastore}
\cvline{\footnotesize Web}{HTML5, CSS, JavaScript, Polymer, AngularJS, React, Node.js}
\cvline{\footnotesize DevOps}{Docker, Kubernetes, Helm, Terraform, GitLab, Github, Travis CI, Google Cloud Platform}

% \cvitem
%     {}
%     {Capable et prêt à monter en compétences sur toutes technologies ou langages.}

%----------------------------------------------------------------------------------------
%	LANGUAGES SECTION
%----------------------------------------------------------------------------------------

\section{Languages}

\cvitemwithcomment
    {}
    {French : Native}
    {}

    \cvitemwithcomment
    {}
    {English : Professional proficiency}
    {990/990  TOEIC (2013)}
\end{document}
